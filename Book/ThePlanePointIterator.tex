\section{PlanePointIterator Class}

The PlanePointIterator is responsible with enumerating the points of a plane object and is very often used in other source files. The following section describes the design of this class.

\subsection {Class declaration}

\begin{lstlisting}
//iterates over the points that make a plane
class PlanePointIterator : public PlanesCommonTools::VectorIterator<PlanesCommonTools::Coordinate2D>
{
    Plane m_plane;
public:
    PlanePointIterator(const Plane& pl);

private:
    void generateList();
};
\end{lstlisting}

The definition above says that the PlanePointIterator is defined as a type of PlanesCommonTools::VectorIterator \textless PlanesCommonTools::Coordinate2D \textgreater which is defined (here I anticipate the VectorIterator class definition which is given later) as an iterator over a list of PlanesCommonTools::Coordinate2D objects. It receives a Plane object with which it generates, with the help of the private function generateList, the list of positions occupied by the plane on the game board. It is a type of Java-like iterator, but this will become apparent only after looking at the VectorIterator class. 

\subsection {Method Implementation}
\begin{lstlisting}
PlanePointIterator::PlanePointIterator(const Plane& pl):
    PlanesCommonTools::VectorIterator<PlanesCommonTools::Coordinate2D>(),
    m_plane(pl)
{
    generateList();
}

//the function that generates the list of points
void PlanePointIterator::generateList()
{

    const PlanesCommonTools::Coordinate2D pointsNorthSouth[] = {PlanesCommonTools::Coordinate2D(0, 0), 
    PlanesCommonTools::Coordinate2D(0, 1), PlanesCommonTools::Coordinate2D(-1, 1),
    PlanesCommonTools::Coordinate2D(1, 1), PlanesCommonTools::Coordinate2D(-2, 1), 
    PlanesCommonTools::Coordinate2D(2, 1), PlanesCommonTools::Coordinate2D(0, 2),
    PlanesCommonTools::Coordinate2D(0, 3), PlanesCommonTools::Coordinate2D(-1, 3),
    PlanesCommonTools::Coordinate2D(1, 3)};

    const PlanesCommonTools::Coordinate2D pointsSouthNorth[] = {PlanesCommonTools::Coordinate2D(0, 0), 
    PlanesCommonTools::Coordinate2D(0, -1), PlanesCommonTools::Coordinate2D(-1, -1),
    PlanesCommonTools::Coordinate2D(1, -1), PlanesCommonTools::Coordinate2D(-2, -1),
    PlanesCommonTools::Coordinate2D(2, -1), PlanesCommonTools::Coordinate2D(0, -2), 
    PlanesCommonTools::Coordinate2D(0, -3), PlanesCommonTools::Coordinate2D(-1, -3),
    PlanesCommonTools::Coordinate2D(1, -3)};

    const PlanesCommonTools::Coordinate2D pointsEastWest[] = {PlanesCommonTools::Coordinate2D(0, 0), 
    PlanesCommonTools::Coordinate2D(1, 0), PlanesCommonTools::Coordinate2D(1, -1),
    PlanesCommonTools::Coordinate2D(1, 1), PlanesCommonTools::Coordinate2D(1, -2), 
    PlanesCommonTools::Coordinate2D(1, 2), PlanesCommonTools::Coordinate2D(2, 0), 
    PlanesCommonTools::Coordinate2D(3, 0), PlanesCommonTools::Coordinate2D(3, -1),
    PlanesCommonTools::Coordinate2D(3, 1)};

    const PlanesCommonTools::Coordinate2D pointsWestEast[] = {PlanesCommonTools::Coordinate2D(0, 0), 
    PlanesCommonTools::Coordinate2D(-1, 0), PlanesCommonTools::Coordinate2D(-1, -1),
    PlanesCommonTools::Coordinate2D(-1, 1), PlanesCommonTools::Coordinate2D(-1, -2), 
    PlanesCommonTools::Coordinate2D(-1, 2), PlanesCommonTools::Coordinate2D(-2, 0), 
    PlanesCommonTools::Coordinate2D(-3, 0), PlanesCommonTools::Coordinate2D(-3, 1),
    PlanesCommonTools::Coordinate2D(-3, -1)};

    const int size = 10;
    for(int i = 0; i < size; ++i)
    {
        switch(m_plane.orientation())
        {
            case Plane::NorthSouth:
                m_internalList.push_back(pointsNorthSouth[i] + m_plane.head());
                break;
            case Plane::SouthNorth:
                m_internalList.push_back(pointsSouthNorth[i] + m_plane.head());
                break;
            case Plane::WestEast:
                m_internalList.push_back(pointsWestEast[i] + m_plane.head());
                break;
            case Plane::EastWest:
                m_internalList.push_back(pointsEastWest[i] + m_plane.head());
                break;
            default:
                ;
        }
    }
}
\end{lstlisting}

PlanePointIterator is an iterator that provides access to the positions occupied by the plane on the game board. It's constructor receives as Parameter an object of the class Plane, which defines the position of the plane's head as well as its orientation. The method generateList() contains the positions of cells belonging to a plane relative to the position of its head for each of the four possible plane orientations. To calculate the positions of the cells corresponding to the plane it chooses the right set of relative positions according to the given plane orientation and then it adds them to the absolute position of the plane head.

To create a game with other forms of ships it is sufficient to reimplement the PlanePointIterator class.

\subsection {Implementation of VectorIterator}

The essence of a PlanePointIterator is captured by the VectorIterator class : a PlanePointIterator is a method (defined in the VectorIterator class) to iterate over a list together with a special list, defined with the method generateList().

\begin{lstlisting}
namespace PlanesCommonTools
{
//defines an iterator over a std::vector

template <class T>
class VectorIterator
{
protected:
    std::vector<T> m_internalList;
    int m_idx;

public:
    //constructor
    VectorIterator();

    //sets the position of the iterator before the first  element
    void reset();
    //during an iteration checks to see if there is a next element
    bool hasNext() const;
    //during an iteration returns the next element
    const T& next();
    //returns number of elements
    int itemNo() const;
};

template <class T>
VectorIterator<T>::VectorIterator()
{
    //generates the list of points
    m_internalList.clear();
    //puts the index one before the first element in the list
    reset();
}

template <class T>
void VectorIterator<T>::reset()
{
    m_idx = -1;
}

//during a point iteration checks to see if there is a next point
template <class T>
bool VectorIterator<T>::hasNext() const
{
    return (m_idx < static_cast<int>(m_internalList.size()) - 1);
}

//during an iteration returns the next point
template <class T>
const T& VectorIterator<T>::next()
{
    return  m_internalList[++m_idx];
}

//returns number of points on the plane
template <class T>
int VectorIterator<T>::itemNo() const
{
    return m_internalList.size();
}

\end{lstlisting}

VectorIterator is a parametrizable class (a so called template class) which has only two member variable, a list (implemented through the STL type std::vector) and an index inside the list. 

\subsection {C++ Concepts}
\subsubsection {Iterators}

Iterators are methods to obtain sequential access to the members of a data container (e.g. vector, list, set). No assumption about the order in which the data is retrieved should be made. In the simplest application scenario an iterator is initialized, an action is performed to the data pointed by the iterator, then the iterator is incremented to the next position and so further until the end of data structure is reached. For a C++ style iterator this would look like this:

\begin{lstlisting}
	//initialization of the iterator at the beginning
	auto it = storage.begin();    
	// read data as long as the data end has not been encountered	
	while (it != storage.end()) {  
		//do something with the data pointed to by the iterator		
		do_something(*it);   
		//go to the next position in the data storage 
   		++it;   
	}
\end{lstlisting}

For a Java style iterator this would be:

\begin{lstlisting}
	//initialize iterator with data	
	auto it(storage);     
	//as long as data is available    
	while (it.hasNext()) { 
		//read the data and jump to the next value
		auto d = it.next();  
		//do something with the data
        do_something(d);  
    }
\end{lstlisting}
\subsubsection {Member variable initialization with an initializer list}

\begin{lstlisting}
//constructor
PlanePointIterator::PlanePointIterator(const Plane& pl):
    PlanesCommonTools::VectorIterator<PlanesCommonTools::Coordinate2D>(),
    m_plane(pl)
{
    generateList();
}
\end{lstlisting}

The constructor of the PlanePointIterator class is interesting. It receives as parameter a Plane object. In its body calls the function generateList() to actually create the points corresponding to the plane. Before the function body it makes other two things. First it calls the default constructor of the basis class : PlanesCommonTools::VectorIterator \textless PlanesCommonTools::Coordinate2D \textgreater() and secondly it initializes its member variable m\textunderscore plane with the constructor parameter. This type of member variable initialization is called initialization with initializer list and it is the member variable initialization method recommended for C++. 

\subsubsection {Templates}

Here below is the definition of class VectorIterator, a so called template class. Actually this is a definition of many classes, each for every parameter type T. T is a parameter itself to the class definition. The programmer specifies the type T only when e.g. an object of this class is instantiated. The ability to use template classes is one of the features specific to C++.

Aside of the parameter T the class definition of VectorIterator is a normal class definition. VectorIterator defines a Java-style iterator over a std::vector. Its member variables are a std::vector with type T elements and an index in the std::vector. The class manages these internal variables and offers functionality to return the element corresponding to the index in the list, to jump to the next element in the list, to reset the list index, and to return the number of elements in the list. Observe how the function next() returns an element of type T.

\begin{lstlisting}
template <class T>
class VectorIterator
{
protected:
    std::vector<T> m_internalList;
    int m_idx;

public:
    //constructor
    VectorIterator();

    //sets the position of the iterator before the first  element
    void reset();
    //during an iteration checks to see if there is a next element
    bool hasNext() const;
    //during an iteration returns the next element
    const T& next();
    //returns number of elements
    int itemNo() const;
};

\end{lstlisting}

\subsubsection {References}

Coming back to the function next() in the VectorIterator class definition. It returns a so called 'const reference' to a value of type T. This is advantageous when the objects of type T are big. The reference, specified here with the const T\&, points only to the data which is saved in the std::vector of the VectorIterator. Had we declared the return type as only T, the next function would have created a copy of the next object in the std::vector, which would have existed twice, once in the std::vector inside the VectorIterator and once as the object created when returning from the next() function.

\subsubsection {Namespaces}
\subsubsection {Class derivation}
\subsubsection {Function parameters and their transmission}
\subsubsection {Constness}

Refering again to the function next() in the VectorIterator class definition, we comment on the const keyword. In this particular case it means that the reference returned by the function next() is const, that is the underlying data may not be changed. The compiler verifies that all operations performed on data returned by the function next() are also const, functions that do not change the object on which they are called. Generally the const property can be applied in C++ to functions, function parameters, function return values, member variables. 
\subsubsection {Constants}