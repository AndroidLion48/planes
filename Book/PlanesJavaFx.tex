\section {PlanesJavaFx}

\subsection{Layout}

The layout of the PlanesJavaFx GUI consists mainly of:

\begin{itemize}
	\item LeftPane - implements the left pane
	\item RightPane - implements the right pane
\end{itemize}

The left pane is a tab widget with three tabs:

\begin{itemize}
	\item a board editing tab, containing the controls to move the planes left, right, downwards and upwards, as well as to rotate them. In this tab there are also a control for toggling the currently selected plane and a button to confirm that the plane positioning is completed.
	\item a game tab, which shows the moves statistics during the game
	\item a start new round tab, showing the global score and a button allowing to start a new round
\end{itemize}

The right pane is a tab widget with two tabs: one is the player game board and the other is the computer game board.

\subsection{Displaying the Game Boards}

\subsubsection {PlaneRoundJavaFx - Use of Java Native Interface}

Because the GUI is implemented with Java and the game engine with C++, we used Java Native Interface to gain access to the C++ game engine from the Java side. 

\begin{lstlisting} [caption={PlaneRoundJavaFX Interface}]

public class PlaneRoundJavaFx {
	static {
	System.loadLibrary("libCommon"); // Load native library 
	}
	
	//creates the PlaneRound object in the game engine
	//must be called a single time	
	public native void createPlanesRound(); 
	
	//show the planes
	public native int getRowNo();
	public native int getColNo();
	public native int getPlaneNo();
	public native int getPlaneSquareType(int i, int j, int isComputer);
	
	//edit the board
	public native int movePlaneLeft(int idx);
	public native int movePlaneRight(int idx);
	public native int movePlaneUpwards(int idx);
	public native int movePlaneDownwards(int idx);
	public native int rotatePlane(int idx);
	public native void doneClicked();
	
	//play the game
	public native void playerGuess(int row, int col);
	public native boolean playerGuess_RoundEnds();
	public native boolean playerGuess_IsPlayerWinner();
	public native boolean playerGuess_ComputerMoveGenerated();
	public native int playerGuess_StatNoPlayerMoves();
	public native int playerGuess_StatNoPlayerHits();
	public native int playerGuess_StatNoPlayerMisses();
	public native int playerGuess_StatNoPlayerDead();
	public native int playerGuess_StatNoPlayerWins();
	public native int playerGuess_StatNoComputerMoves();
	public native int playerGuess_StatNoComputerHits();
	public native int playerGuess_StatNoComputerMisses();
	public native int playerGuess_StatNoComputerDead();
	public native int playerGuess_StatNoComputerWins();
	
	public native void roundEnds();
	public native void initRound();
	
	//show the guesses
	public native int getPlayerGuessesNo();
	public native int getPlayerGuessRow(int idx);
	public native int getPlayerGuessCol(int idx);
	public native int getPlayerGuessType(int idx);
	
	public native int getComputerGuessesNo();
	public native int getComputerGuessRow(int idx);
	public native int getComputerGuessCol(int idx);
	public native int getComputerGuessType(int idx);	
}

\end{lstlisting}

The class PlaneRoundJavaFx loads the libCommon library and defines a series of methods which are declared with the keyword native, that is they are implemented in a C/C++ library. The native methods represent the only gate of access of the Java GUI to the Planes game engine. They do the following:

\begin{itemize}
	\item createPlanesRound() - initialize the game engine
	
	\item getRowNo() - gets the size of the game board
	\item getColNo() - gets the size of the game board
	\item getPlaneNo() - gets the plane number 
	\item getPlaneSquareType(int i, int j, int isComputer) - for a square on the game board returns what it contains: a plane head, plane, not plane, game board, outside the game board
	
	\item movePlaneLeft() - repositions the selected plane to the left
	\item movePlaneRight() - repositions the selected plane to the right
	\item movePlaneUpwards() - repositions the selected plane upwards
	\item movePlaneDownwards() - repositions the selected plane downwards
	\item rotatePlane() - rotates 90 degrees the selected plane
	\item doneClicked() - end board editing phase 
	
	\item getPlayerGuessesNo() - how many guesses has the player made
	\item getPlayerGuessRow() - coordinate of the desired player guess
	\item getPlayerGuessCol() - coordinate of the desired player guess
	\item getPlayerGuessType() - result of the desired player guess
	
	\item getComputerGuessesNo() - how many guesses has the computer made
	\item getComputerGuessRow() - coordinate of the desired computer guess
	\item getComputerGuessCol() -  coordinate of the desired computer guess
	\item getComputerGuessType() - 	result of the desired computer guess
	
	\item playerGuess(int row, int col) - communicate a guess of the player to the game engine
	\item playerGuess\_RoundEnds() - does the round end
	\item playerGuess\_IsPlayerWinner() - who won
	\item playerGuess\_ComputerMoveGenerated() - if a computer move was generated
	\item playerGuess\_StatNoPlayerMoves() - statistics about the player's moves
	\item playerGuess\_StatNoPlayerHits() - statistics about the player's moves
	\item playerGuess\_StatNoPlayerMisses() - statistics about the player's moves
	\item playerGuess\_StatNoPlayerDead() - statistics about the player's moves
	\item playerGuess\_StatNoPlayerWins() - number of wins for the player
	\item playerGuess\_StatNoComputerMoves() - statistics about the computers's moves
	\item playerGuess\_StatNoComputerHits() - statistics about the computers's moves
	\item playerGuess\_StatNoComputerMisses() - statistics about the computers's moves
	\item playerGuess\_StatNoComputerDead() - statistics about the computers's moves
	\item playerGuess\_StatNoComputerWins() - number of wins for the computer
	
	\item roundEnds() - do what is required when the round ends
	\item initRound() - do what is required to initialize a new round

\end{itemize}

Corresponding to this Java class, a C++ implementation of the required functionality was created in the game engine library. The C++ implementation is a wrapper around the PlaneRound game controller where almost all functions are single method calls of the PlaneRound object. The header of the implementation is created automatically with the javac tool. An excerpt is given in the following listing (TODO applies this only to Windows):

\begin{lstlisting} [caption={C++ Implementation for PlaneRoundJavaFx}]
/* DO NOT EDIT THIS FILE - it is machine generated */
#include <jni.h>
/* Header for class com_planes_javafx_PlaneRoundJavaFx */

#ifndef _Included_com_planes_javafx_PlaneRoundJavaFx
#define _Included_com_planes_javafx_PlaneRoundJavaFx
#ifdef __cplusplus
extern "C" {
#endif
	/*
	* Class:     com_planes_javafx_PlaneRoundJavaFx
	* Method:    createPlanesRound
	* Signature: ()V
	*/
	JNIEXPORT void JNICALL Java_com_planes_javafx_PlaneRoundJavaFx_createPlanesRound
	(JNIEnv *, jobject);
	
	.....

#ifdef __cplusplus
}
#endif
#endif

\end{lstlisting}

In the .cpp file all the functions work with one or more of these 3 global objects:

\begin{lstlisting}[caption={Global Variables}\label{Global_Variable_Definitions}] 
PlaneRound* global_Round = nullptr;
GuessPoint::Type global_Guess_Result = GuessPoint::Miss;
PlayerGuessReaction global_Player_Guess_Reaction;
\end{lstlisting}

global\_Round is the game controller, global\_Guess\_Result is the result of the evaluation of the last player guess, global\_Player\_Guess\_Reaction is the response of the game engine to the last player guess. 

global\_Round is created with the function createPlanesRound() from PlaneRoundJavaFx which corresponds to the function\\ Java\_com\_planes\_javafx\_PlaneRoundJavaFx\_createPlanesRound() in the C++ implementation file. global\_Guess\_Result and global\_Player\_Guess\_Reaction are obtained from the following function call:

\begin{lstlisting}
global_Round->playerGuessIncomplete(int(row), int(col), global_Guess_Result, global_Player_Guess_Reaction);
\end{lstlisting}

The function playerGuessIncomplete is defined in the PlaneRound class (\ref{Game_Controller}).

\subsubsection {The BoardPane Class}

The computer's game board and the player's game board are parts of the right pane and are displayed using the functionality offered by the PlaneRoundJavaFx object. The functionality is implemented in the class BoardPane.

The member variables are as follows:

\begin{lstlisting} [caption={PlaneBoard's Member Variable}]

	private Map<PositionBoardPane, Canvas> m_GridSquares;
	private PlaneRoundJavaFx m_PlaneRound;
	private RightPane m_RightPane;
	private int m_Padding = 3;
	private boolean m_IsComputer = false;
	private int m_MinPlaneBodyColor = 0;
	private int m_MaxPlaneBodyColor = 200;	
	private GameStages m_CurStage = GameStages.BoardEditing;
	private int m_SelectedPlane = 0;
	private EventHandler<MouseEvent> m_ClickedHandler;
	private Text m_AnimatedText;
	private GridPane m_GridPane;

	private int m_GRows = 10;
	private int m_GCols = 10;
	private int m_PlaneNo = 3;
	private int m_ColorStep = 50;

\end{lstlisting}

The gameboard is a rectangle of squares with a padding, allowing the rotation of planes when they are close to the boundary of the board. How big is the padding is defined by the member variable m\_Padding. Whether the board belongs to the player or to the computer is defined by the boolean m\_IsComputer. For editing the game board planes must be selected in order to move them. The currently selected plane is defined with the variable m\_SelectedPlane. Each of the planes on the game board is displayed in a gray tone, which is defined by the maximum and minimum gray tone as well as by the number of planes. The maximum and minimum gray tone levels are defined by the variables m\_MinPlaneBodyColor and m\_MaxPlaneBodyColor. The current stage of the game (game, board editing, game not started yet) is defined by the variable m\_CurStage.

What is displayed in each of the cells of the game board is defined with the map m\_GridSquares, these are objects of the Canvas class, which is a graphical object in which one can draw. The layout of the board is defined with the m\_GridPane variable. What happens when the user clicks on board cell is defined with the event handler m\_ClickedHandler. 

At the end of a round an animated text is displayed in the computer's board. This text is defined with the variable m\_AnimatedText. 

A reference to the right pane is saved in the variable m\_RightPane such that  methods of this class can be directly called. Finally the game engine is saved in the variable m\_PlaneRound.

The color of the board cell is computed with the following function:

\begin{lstlisting} [caption={Computing Grid Board Colors}]

public Color computeSquareBackgroundColor(int i, int j) {
	Color squareColor = null;
	
	if (i < m_Padding || i >= m_GRows + m_Padding || j < m_Padding || j >= m_GCols + m_Padding) {
		squareColor = Color.YELLOW;
	} else {
		squareColor = Color.AQUA;
	}
	
	if (!m_IsComputer || (m_IsComputer && m_CurStage == GameStages.GameNotStarted)) {
		int type = m_PlaneRound.getPlaneSquareType(i - m_Padding, j - m_Padding, m_IsComputer ? 1 : 0);
		switch (type) {					
		//intersecting planes
		case -1:
			squareColor = Color.RED;
			break;								
		//plane head
		case -2:
			squareColor = Color.GREEN;
			break;			
		//not a plane	
		case 0:
			break;					
		//plane but not plane head
		default:
			if ((type - 1) == m_SelectedPlane) {
				squareColor = Color.BLUE;
			} else {
				int grayCol = m_MinPlaneBodyColor + type * m_ColorStep;
				squareColor = Color.rgb(grayCol, grayCol, grayCol);
			}
			break;						
		}
	}
	return squareColor;
}

\end{lstlisting}

\subsection{C++ Concepts}

\subsubsection{Global Variables} 

The three variables defined in \ref{Global_Variable_Definitions} are global variables, that is they are defined outside the body of any function. As opposed to local variables, which are defined inside the body of functions, the lifetime of these variables is the entire lifetime of the program. They also have file scope, that means they are accessible from functions defined in the same file as the global variables.

\subsection{JavaFx Concepts}

\subsubsection{Layouts}

The following layout of JavaFx were used: GridPane, a grid like layout, TabPane, a window with more tabs.

\subsubsection{Properties and Property Binding}

Properties in JavaFx are data containers which can signal to other components when the associated values change. Properties can be interconnected one to another through a mechanism called binding. That allows for example the resizing of the game board grid squares when the game window is resized.

TODO: Example

\subsection {Java Concepts}

\subsubsection{Access Specifiers}

Access specifiers in Java are the same as in C++ (TODO: reference) except that they are placed in front of each member variable or member function.

\subsubsection {Event Handlers}

Event handlers in Java are functions that are automatically called when an event occurs.

TODO: Example