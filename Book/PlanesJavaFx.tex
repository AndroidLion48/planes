\section {PlanesJavaFx}

The main components of PlanesJavaFx GUI are:

\begin{itemize}
	\item LeftPane - implements the left pane
	\item RightPane - implements the right pane
	\item PlanesJavaFxApplication  - the application class as needed by JavaFx
	\item PlaneRoundJavaFx - the game engine object
\end{itemize}


\subsection {PlaneRoundJavaFx - Use of Java Native Interface}

Because the GUI is implemented with Java and the game engine with C++, we used Java Native Interface to gain access to C++ code from the Java side. 

\begin{lstlisting} 

public class PlaneRoundJavaFx {
	static {
	System.loadLibrary("libCommon"); // Load native library 
	}
	
	//creates the PlaneRound object in the game engine
	//must be called a single time	
	public native void createPlanesRound(); 
	
	//show the planes
	public native int getRowNo();
	public native int getColNo();
	public native int getPlaneNo();
	public native int getPlaneSquareType(int i, int j, int isComputer);
	
	//edit the board
	public native void movePlaneLeft(int idx);
	public native void movePlaneRight(int idx);
	public native void movePlaneUpwards(int idx);
	public native void movePlaneDownwards(int idx);
	public native void rotatePlane(int idx);
	public native void doneClicked();
	
	//play the game
	public native void playerGuess(int row, int col);
	public native boolean playerGuess_RoundEnds();
	public native boolean playerGuess_IsPlayerWinner();
	public native boolean playerGuess_ComputerMoveGenerated();
	public native int playerGuess_StatNoPlayerMoves();
	public native int playerGuess_StatNoPlayerHits();
	public native int playerGuess_StatNoPlayerMisses();
	public native int playerGuess_StatNoPlayerDead();
	public native int playerGuess_StatNoPlayerWins();
	public native int playerGuess_StatNoComputerMoves();
	public native int playerGuess_StatNoComputerHits();
	public native int playerGuess_StatNoComputerMisses();
	public native int playerGuess_StatNoComputerDead();
	public native int playerGuess_StatNoComputerWins();
	
	public native void roundEnds();
	public native void initRound();
	
	//show the guesses
	public native int getPlayerGuessesNo();
	public native int getPlayerGuessRow(int idx);
	public native int getPlayerGuessCol(int idx);
	public native int getPlayerGuessType(int idx);
	
	public native int getComputerGuessesNo();
	public native int getComputerGuessRow(int idx);
	public native int getComputerGuessCol(int idx);
	public native int getComputerGuessType(int idx);	
}


\end{lstlisting}

The class PlaneRoundJavaFx loads the libCommon library and defines a series of methods which are declared with the keyword native, that is they are implemented in a C/C++ library. The native methods represent the only method of access of the Java GUI to the Planes game engine. They do the following:

\begin{itemize}
	\item createPlanesRound() - initialize the game engine
	
	\item getRowNo() - gets the size of the game board
	\item getColNo() - gets the size of the game board
	\item getPlaneNo() - gets the plane number 
	\item getPlaneSquareType(int i, int j, int isComputer) - for a square on the game board returns what it contains: a plane head, plane, not plane, game board, outside the game board
	
	\item movePlaneLeft() - repositions the selected plane to the left
	\item movePlaneRight() - repositions the selected plane to the right
	\item movePlaneUpwards() - repositions the selected plane upwards
	\item movePlaneDownwards() - repositions the selected plane downwards
	\item rotatePlane() - rotates 90 degrees the selected plane
	\item doneClicked() - end board editing phase 
	
	\item getPlayerGuessesNo() - how many guesses has the player made
	\item getPlayerGuessRow() - coordinate of the desired player guess
	\item getPlayerGuessCol() - coordinate of the desired player guess
	\item getPlayerGuessType() - result of the desired player guess
	
	\item getComputerGuessesNo() - how many guesses has the computer made
	\item getComputerGuessRow() - coordinate of the desired computer guess
	\item getComputerGuessCol() -  coordinate of the desired computer guess
	\item getComputerGuessType() - 	result of the desired computer guess
	
	\item playerGuess(int row, int col) - communicate a guess of the player to the game engine
	\item playerGuess\_RoundEnds() - does the round end
	\item playerGuess\_IsPlayerWinner() - who won
	\item playerGuess\_ComputerMoveGenerated() - if a computer move was generated
	\item playerGuess\_StatNoPlayerMoves() - statistics about the player's moves
	\item playerGuess\_StatNoPlayerHits() - statistics about the player's moves
	\item playerGuess\_StatNoPlayerMisses() - statistics about the player's moves
	\item playerGuess\_StatNoPlayerDead() - statistics about the player's moves
	\item playerGuess\_StatNoPlayerWins() - number of wins for the player
	\item playerGuess\_StatNoComputerMoves() - statistics about the computers's moves
	\item playerGuess\_StatNoComputerHits() - statistics about the computers's moves
	\item playerGuess\_StatNoComputerMisses() - statistics about the computers's moves
	\item playerGuess\_StatNoComputerDead() - statistics about the computers's moves
	\item playerGuess\_StatNoComputerWins() - number of wins for the computer
	
	\item roundEnds() - do what is required when the round ends
	\item initRound() - do what is required to initialize a new round

\end{itemize}

Corresponding to this Java class, a C++ implementation of the required functionality was created in the game engine library. The C++ implementation is a wrapper around the PlaneRound game controller where almost all functions are single method calls of the PlaneRound object. The header of the implementation is created automatically with the javac tool. An excerpt is given in the following listing:

\begin{lstlisting} 
/* DO NOT EDIT THIS FILE - it is machine generated */
#include <jni.h>
/* Header for class com_planes_javafx_PlaneRoundJavaFx */

#ifndef _Included_com_planes_javafx_PlaneRoundJavaFx
#define _Included_com_planes_javafx_PlaneRoundJavaFx
#ifdef __cplusplus
extern "C" {
#endif
/*
* Class:     com_planes_javafx_PlaneRoundJavaFx
* Method:    createPlanesRound
* Signature: ()V
*/
JNIEXPORT void JNICALL Java_com_planes_javafx_PlaneRoundJavaFx_createPlanesRound
(JNIEnv *, jobject);

.....

#ifdef __cplusplus
}
#endif
#endif

\end{lstlisting}

In the .cpp file all the functions work with 3 global objects:

\begin{lstlisting}
PlaneRound* global_Round = nullptr;
GuessPoint::Type global_Guess_Result = GuessPoint::Miss;
PlayerGuessReaction global_Player_Guess_Reaction;
\end{lstlisting}

global\_Round is the game controller, global\_Guess\_Result is the result of the evaluation of the last player guess, global\_Player\_Guess\_Reaction is the response of the game engine to the last player guess. (TODO: Reference to game engine section)

global\_Round is created with the function createPlanesRound() from PlaneRoundJavaFx which corresponds to the function\\ Java\_com\_planes\_javafx\_PlaneRoundJavaFx\_createPlanesRound() in the C++ implementation file. global\_Guess\_Result and global\_Player\_Guess\_Reaction are obtained from the following function call:

\begin{lstlisting}
global_Round->playerGuessIncomplete(int(row), int(col), global_Guess_Result, global_Player_Guess_Reaction);
\end{lstlisting}

The function playerGuessIncomplete is defined in the PlaneRound class (TODO see reference).